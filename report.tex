\documentclass[12pt]{article}
\usepackage{fullpage}
\usepackage{color}
\usepackage{graphicx}
\usepackage{amsmath}
\usepackage{amsfonts}
\usepackage{comment}
\usepackage{verbatim}

\title{XINU Independent Study Report}
\author{David Larsen}
\date{\today}

\newcommand{\TODO}[1]{{\huge\color{red}TODO \Large #1}}
\newcommand{\code}[1]{{\tt #1}}

\begin{document}

\maketitle

\tableofcontents

\section{ARM Core Interrupt Handler}

\TODO{}

\section{XINU's platform configuration compiler}

\subsection{Adding devices to a XINU platform}

Marqueette University's version of XINU includes a small compiler for
transforming a description of the hardware for a specific plaform into header
files and a routine that populates the device table at startup.

So, the description:

\begin{verbatim}
uart:
        on HARDWARE -i uartInit     -o ionull        -c ionull
                    -r uartRead    -g uartGetc       -w uartWrite 
                    -n ionull      -p uartPutc       -intr uartInterrupt

/* Two uarts on the versatile */
SERIAL0   is uart     on HARDWARE csr 0x101F1000 irq 12
SERIAL1   is uart     on HARDWARE csr 0x101F2000 irq 13
\end{verbatim}

will generate a device entry like this:

\begin{verbatim}
typedef struct dentry
{
        int     num;
        int     minor;
        char    *name;
        devcall (*init)(struct dentry *);
        devcall (*open)(struct dentry *, ...);
        devcall (*close)(struct dentry *);
        devcall (*read)(struct dentry *, void *, uint);
        devcall (*write)(struct dentry *, void *, uint);
        devcall (*seek)(struct dentry *, long);
        devcall (*getc)(struct dentry *);
        devcall (*putc)(struct dentry *, char);
        devcall (*control)(struct dentry *, int, long, long);
        void    *csr;
        void    (*intr)(void);
        uchar   irq;
} device;

[...]

	{ 0, 0, "SERIAL0",
	  (void *)uartInit, (void *)ionull, (void *)ionull,
	  (void *)uartRead, (void *)uartWrite, (void *)ioerr,
	  (void *)uartGetc, (void *)uartPutc, (void *)ionull,
	  (void *)0x101f1000, (void *)uartInterrupt, 12 },
\end{verbatim}

So, a routine like \code{puts()} only needs to call
\code{dentry[\$DEVICE].write()}, without caring what kind of device it is, or
how it's implemented.

\subsection{Fixing changes in Flex}

\TODO{}

\section{QEMU}

\TODO{}

\subsection{QEMU's broken VIC}

\TODO{}

\end{document}
